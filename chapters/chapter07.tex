\chapter{Root Filesystem and Init Systems}

\section{The Role of Rootfs}
The Kernel mounts the Root Filesystem (\texttt{/}) to access applications, libraries, and configuration files. It is the first user-space component.

\section{Directory Hierarchy (FHS)}
Embedded Linux generally follows the Filesystem Hierarchy Standard:
\begin{itemize}
    \item \texttt{/bin, /sbin}: Essential binaries (sh, mount, init).
    \item \texttt{/etc}: System configuration.
    \item \texttt{/lib}: Kernel modules and shared libraries.
    \item \texttt{/dev}: Device nodes (populated by devtmpfs).
    \item \texttt{/proc, /sys}: Virtual filesystems for kernel interface.
    \item \texttt{/var}: Volatile data (logs, locks).
\end{itemize}

\section{Init Systems}
After mounting rootfs, the kernel executes \texttt{/sbin/init} (PID 1).

\subsection{SysVinit (The Classic)}
\begin{itemize}
    \item \textbf{Pros}: Simple shell scripts, deterministic, lightweight.
    \item \textbf{Cons}: Serial execution (slow boot), no supervision (if a service dies, it stays dead).
    \item \textbf{Config}: \texttt{/etc/inittab}, \texttt{/etc/init.d/}.
\end{itemize}

\subsection{systemd (The Modern Standard)}
\begin{itemize}
    \item \textbf{Pros}: Parallel boot, service supervision, socket activation, powerful logging (journald).
    \item \textbf{Cons}: Large footprint, complex.
    \item \textbf{Config}: Unit files in \texttt{/lib/systemd/system/}.
\end{itemize}

\subsection{BusyBox Init}
Extremely minimal. Controlled by a single \texttt{/etc/inittab}. Ideal for tiny systems (<16MB storage).

\section{BusyBox: The Swiss Army Knife}
BusyBox combines hundreds of common utilities (\texttt{ls}, \texttt{cp}, \texttt{sh}, \texttt{mount}) into a single binary.
\begin{itemize}
    \item \textbf{Symlinks}: \texttt{/bin/ls} is a symlink to \texttt{/bin/busybox}.
    \item \textbf{Size}: < 1MB typically.
    \item \textbf{Config}: \texttt{make menuconfig} to select enabled applets.
\end{itemize}

\section{Pseudo-Filesystems}
These are critical for system operation:
\begin{itemize}
    \item \textbf{/proc}: Process information. \texttt{mount -t proc none /proc}
    \item \textbf{/sys}: Kernel object model (drivers/devices). \texttt{mount -t sysfs none /sys}
    \item \textbf{/dev}: Device nodes. Managed by \texttt{devtmpfs} (kernel) and \texttt{udev/mdev} (userspace).
\end{itemize}

\section{Hands-On: Switching Init Systems in Yocto}
In \texttt{local.conf}, you can switch between SysVinit and systemd easily.

\textbf{For systemd}:
\begin{lstlisting}
DISTRO_FEATURES:append = " systemd"
VIRTUAL-RUNTIME_init_manager = "systemd"
DISTRO_FEATURES_BACKFILL_CONSIDERED = "sysvinit"
\end{lstlisting}

\textbf{For SysVinit}:
\begin{lstlisting}
DISTRO_FEATURES:append = " sysvinit"
DISTRO_FEATURES:remove = "systemd"
VIRTUAL-RUNTIME_init_manager = "sysvinit"
\end{lstlisting}
