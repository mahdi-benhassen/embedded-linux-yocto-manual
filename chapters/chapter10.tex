 \chapter{Yocto Build Setup}
 \section{Obtain Sources}
 Clone the Poky repository and any needed layers. Choose a release branch appropriate for your product.
 \begin{lstlisting}[language=bash]
 mkdir -p ~/yocto
 cd ~/yocto
 git clone https://git.yoctoproject.org/poky
 cd poky
 git checkout kirkstone
 \end{lstlisting}
 \section{Initialize Build Directory}
 Source the environment script to create and enter a build directory.
 \begin{lstlisting}[language=bash]
 source oe-init-build-env
 \end{lstlisting}
 \section{Configure Layers}
 Edit conf/bblayers.conf to add external layers and conf/local.conf for machine and image customization. Verify layer compatibility with the selected release.
 \section{First Build}
 Build a minimal image to confirm the environment and dependencies.
 \begin{lstlisting}[language=bash]
 bitbake core-image-minimal
 \end{lstlisting}
 \section{Artifacts}
 Inspect tmp/deploy/images for generated binaries. Use wic to create bootable images when supported by the machine.
