\chapter{Yocto Build Setup}

\section{Prerequisites}
Before starting, ensure your host machine (Ubuntu 20.04/22.04 LTS is recommended) has the necessary dependencies.
\begin{lstlisting}[language=bash]
sudo apt update && sudo apt install -y gawk wget git diffstat unzip texinfo \
gcc build-essential chrpath socat cpio python3 python3-pip python3-pexpect \
xz-utils debianutils iputils-ping python3-git python3-jinja2 libegl1-mesa \
libsdl1.2-dev pylint3 xterm zstd liblz4-tool
\end{lstlisting}

\section{Obtaining Poky}
Clone the reference distribution, Poky. We will use the \textbf{Kirkstone} release (LTS).
\begin{lstlisting}[language=bash]
mkdir -p ~/yocto
cd ~/yocto
git clone -b kirkstone https://git.yoctoproject.org/poky
\end{lstlisting}

\section{Initializing the Environment}
Source the build environment script. This sets up your \texttt{PATH} and creates the build directory.
\begin{lstlisting}[language=bash]
source poky/oe-init-build-env build-qemu
\end{lstlisting}
You are now in the \texttt{build-qemu/} directory.

\section{Configuring local.conf}
The \texttt{conf/local.conf} file controls the build settings. Edit it to optimize performance and disk usage.

\subsection{Essential Variables}
\begin{itemize}
    \item \texttt{MACHINE}: The target hardware. Defaults to \texttt{qemux86-64}. For ARM emulation, use:
    \begin{lstlisting}
MACHINE ?= "qemuarm"
    \end{lstlisting}
    \item \texttt{DL\_DIR}: Where to store downloaded sources. \textbf{Move this outside the build dir} to share it.
    \begin{lstlisting}
DL_DIR ?= "${TOPDIR}/../downloads"
    \end{lstlisting}
    \item \texttt{SSTATE\_DIR}: Where to store shared state cache.
    \begin{lstlisting}
SSTATE_DIR ?= "${TOPDIR}/../sstate-cache"
    \end{lstlisting}
    \item \texttt{BB\_NUMBER\_THREADS} \& \texttt{PARALLEL\_MAKE}: Set these to your CPU core count (e.g., 8).
    \item \texttt{PACKAGE\_CLASSES}: Package format (rpm, deb, ipk).
    \begin{lstlisting}
PACKAGE_CLASSES ?= "package_rpm"
    \end{lstlisting}
    \item \texttt{EXTRA\_IMAGE\_FEATURES}: Enable debugging tweaks (not for production!).
    \begin{lstlisting}
EXTRA_IMAGE_FEATURES ?= "debug-tweaks"
    \end{lstlisting}
\end{itemize}

\section{Building an Image}
Build the minimal image to verify setup:
\begin{lstlisting}[language=bash]
bitbake core-image-minimal
\end{lstlisting}
This will take time (fetching sources, building toolchain, compiling kernel/packages).

\section{Running with QEMU}
Once built, run the image in the emulator:
\begin{lstlisting}[language=bash]
runqemu qemuarm nographic
\end{lstlisting}
Login as \texttt{root} (no password). To exit QEMU, press \texttt{Ctrl+A} then \texttt{X}.

\section{Troubleshooting}
\begin{itemize}
    \item \textbf{Fetch Failures}: Check your proxy settings or try a different mirror.
    \item \textbf{Disk Space}: Yocto needs 50GB+. If you run out, delete \texttt{tmp/}.
    \item \textbf{Missing Headers}: Ensure host dependencies are installed.
\end{itemize}
