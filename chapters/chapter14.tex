\chapter{Kernel and BSP with Yocto}

\section{Building the Kernel}
Yocto provides the \texttt{kernel} class to handle building Linux. The standard recipe is \texttt{linux-yocto}, but most BSPs provide their own (e.g., \texttt{linux-raspberrypi}, \texttt{linux-imx}).

\section{Kernel Configuration}
Do \textbf{not} just run \texttt{menuconfig}, save the \texttt{.config}, and check it into git. That is hard to maintain. Instead, use \textbf{Configuration Fragments}.

\subsection{Using Fragments (.cfg)}
A fragment contains only the \emph{changes} from the default configuration.
1. Create \texttt{files/wifi.cfg}:
\begin{lstlisting}
CONFIG_WLAN=y
CONFIG_IWlwifi=m
\end{lstlisting}

2. Add it to the kernel recipe via \texttt{.bbappend}:
\begin{lstlisting}
SRC_URI += "file://wifi.cfg"
\end{lstlisting}

\subsection{Running Menuconfig}
To inspect or create config changes:
\begin{lstlisting}[language=bash]
bitbake -c menuconfig virtual/kernel
\end{lstlisting}
After saving, run:
\begin{lstlisting}[language=bash]
bitbake -c diffconfig virtual/kernel
\end{lstlisting}
This outputs a \texttt{fragment.cfg} containing only your changes.

\section{Device Tree Management}
The Device Tree (DT) describes the hardware.

\subsection{Specifying DTB in Recipe}
Set the \texttt{KERNEL\_DEVICETREE} variable in your machine config or local.conf:
\begin{lstlisting}
KERNEL_DEVICETREE = "am335x-boneblack.dtb"
\end{lstlisting}

\subsection{Device Tree Overlays}
Overlays modify the DT at runtime (e.g., for a HAT or cape).
To add an overlay source (\texttt{.dts}) to the compilation:
1. Add it to \texttt{SRC\_URI}.
2. Ensure it compiles to \texttt{.dtbo}.

\section{Creating a BSP Layer}
A Board Support Package (BSP) layer contains machine configurations.

\subsection{Structure}
\begin{itemize}
    \item \texttt{meta-myboard/conf/machine/myboard.conf}: The machine definition.
    \item \texttt{meta-myboard/recipes-kernel/linux/}: Kernel recipes.
    \item \texttt{meta-myboard/recipes-bsp/u-boot/}: Bootloader recipes.
\end{itemize}

\subsection{Machine Configuration Example}
\texttt{conf/machine/myboard.conf}:
\begin{lstlisting}
require conf/machine/include/soc-family.inc

PREFERRED_PROVIDER_virtual/kernel = "linux-myboard"
PREFERRED_PROVIDER_virtual/bootloader = "u-boot-myboard"

MACHINE_FEATURES = "ext2 serial usbhost vfat wifi"

KERNEL_IMAGETYPE = "zImage"
KERNEL_DEVICETREE = "myboard.dtb"

SERIAL_CONSOLES = "115200;ttyS0"
\end{lstlisting}

\section{Firmware Loading}
Many drivers require binary firmware (WiFi, GPU).
1. Add \texttt{linux-firmware} to your image.
2. If the firmware is proprietary, create a recipe that installs the \texttt{.bin} file to \texttt{/lib/firmware}.
\begin{lstlisting}
FILES:${PN} += "/lib/firmware/my-device.bin"
\end{lstlisting}
