\chapter{Device Tree and Drivers}

\section{What is a Device Tree?}
On x86 (PCs), hardware is discoverable (PCIe enumeration, USB, ACPI). On embedded ARM, the kernel has no way to know that "a UART is at address 0x48020000" unless we tell it.
The \textbf{Device Tree (DT)} is a data structure describing the hardware layout.

\section{DTS Syntax}
Device Tree Source (.dts) is a human-readable text file.

\begin{lstlisting}
/dts-v1/;
#include "imx6ull.dtsi"

/ {
    model = "My Custom Board";
    compatible = "myvendor,myboard", "fsl,imx6ull";

    chosen {
        bootargs = "console=ttymxc0,115200";
    };

    leds {
        compatible = "gpio-leds";
        pinctrl-names = "default";
        pinctrl-0 = <&pinctrl_gpio_leds>;

        status {
            label = "status_led";
            gpios = <&gpio1 3 GPIO_ACTIVE_LOW>;
            linux,default-trigger = "heartbeat";
        };
    };
};

&uart1 {
    pinctrl-names = "default";
    pinctrl-0 = <&pinctrl_uart1>;
    status = "okay";
};
\end{lstlisting}

\section{Key Properties}
\begin{itemize}
    \item \texttt{compatible}: The most important property. It matches the node to a specific kernel driver.
    \item \texttt{reg}: Physical address and size of memory-mapped registers.
    \item \texttt{interrupts}: IRQ lines connected to the interrupt controller.
    \item \texttt{status}: Set to \texttt{"okay"} to enable, or \texttt{"disabled"} to ignore.
    \item \texttt{phandle} (\texttt{\&label}): A reference to another node (e.g., \texttt{\&uart1}).
\end{itemize}

\section{Compiling the Device Tree}
The Device Tree Compiler (\texttt{dtc}) converts \texttt{.dts} (text) to \texttt{.dtb} (binary).
\begin{lstlisting}[language=bash]
# Compile
dtc -I dts -O dtb -o myboard.dtb myboard.dts

# Reverse compile (Binary to Text - useful for debugging!)
dtc -I dtb -O dts myboard.dtb
\end{lstlisting}

\section{Interacting with Hardware from Userspace}
Before writing a kernel driver, try to control hardware from userspace.

\subsection{GPIO (New Way: libgpiod)}
Do not use the deprecated \texttt{/sys/class/gpio}. Use \texttt{gpiod} tools.
\begin{lstlisting}[language=bash]
gpiodetect          # List chips
gpioinfo            # List lines
gpioset 0 3=1       # Set chip 0, line 3 to High
gpioget 0 3         # Read chip 0, line 3
\end{lstlisting}

\subsection{I2C}
Use \texttt{i2c-tools}.
\begin{lstlisting}[language=bash]
i2cdetect -y 1      # Scan bus 1
i2cget -y 1 0x50 0x00  # Read byte from device 0x50
\end{lstlisting}

\subsection{LEDs}
Controlled via sysfs.
\begin{lstlisting}[language=bash]
echo 1 > /sys/class/leds/status_led/brightness
echo timer > /sys/class/leds/status_led/trigger
\end{lstlisting}
