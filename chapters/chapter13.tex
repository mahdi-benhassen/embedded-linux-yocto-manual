\chapter{Building Applications with Yocto}

\section{Overview}
Yocto is not just for building the OS; it is also a build system for your applications. By creating recipes, you ensure your app is cross-compiled against the correct libraries and packaged automatically.

\section{Build Systems}
Yocto provides "classes" to handle common build systems.

\subsection{CMake}
Inherit \texttt{cmake} class. Yocto automatically passes the toolchain file.

\begin{lstlisting}
SUMMARY = "My CMake App"
LICENSE = "MIT"
LIC_FILES_CHKSUM = "file://LICENSE;md5=..."
SRC_URI = "git://github.com/me/app.git;branch=main"
S = "${WORKDIR}/git"

inherit cmake

# Pass extra arguments to cmake
EXTRA_OECMAKE = "-DENABLE_FEATURE_X=ON"
\end{lstlisting}

\subsection{Autotools (GNU Build System)}
Inherit \texttt{autotools} class. Handles \texttt{./configure}, \texttt{make}, \texttt{make install}.

\begin{lstlisting}
inherit autotools pkgconfig

# If the source doesn't have a configure script yet (needs autoreconf)
DEPENDS += "autoconf-archive"
\end{lstlisting}

\subsection{Python}
Inherit \texttt{setuptools3} (for setup.py) or \texttt{python\_flit\_core} (for pyproject.toml).

\begin{lstlisting}
inherit setuptools3
RDEPENDS:${PN} += "python3-requests"
\end{lstlisting}

\section{Installing Systemd Services}
To start your application at boot, you need a systemd unit file.

1. Create the unit file \texttt{files/myapp.service}:
\begin{lstlisting}
[Unit]
Description=My Custom Application
After=network.target

[Service]
ExecStart=/usr/bin/myapp
Restart=always

[Install]
WantedBy=multi-user.target
\end{lstlisting}

2. Update the recipe:
\begin{lstlisting}
inherit systemd

SYSTEMD_SERVICE:${PN} = "myapp.service"

SRC_URI += "file://myapp.service"

do_install:append() {
    install -d ${D}${systemd_system_unitdir}
    install -m 0644 ${WORKDIR}/myapp.service ${D}${systemd_system_unitdir}
}
\end{lstlisting}

\section{Managing Users and Groups}
If your application needs to run as a specific user, do not use \texttt{useradd} in shell scripts. Use \texttt{useradd.bbclass}.

\begin{lstlisting}
inherit useradd

USERADD_PACKAGES = "${PN}"
USERADD_PARAM:${PN} = "-u 1200 -d /home/myuser -r -s /bin/false myuser"
GROUPADD_PARAM:${PN} = "-g 1200 mygroup"
\end{lstlisting}

\section{Package Groups}
Instead of adding individual recipes to \texttt{local.conf}, create a "Package Group" recipe to bundle them.

Create \texttt{recipes-core/packagegroups/packagegroup-my-product.bb}:
\begin{lstlisting}
SUMMARY = "My Product Packages"
inherit packagegroup

RDEPENDS:${PN} = "\
    myapp \
    networkmanager \
    openssh \
    python3-numpy \
"
\end{lstlisting}

Then add \texttt{packagegroup-my-product} to your image recipe.
