\chapter{Layers and Customization}

\section{The Layer Model}
Yocto's modularity comes from "layers". A layer is a Git repository containing recipes, configuration, and classes. Layers can stack on top of each other.
\begin{itemize}
    \item \textbf{OE-Core (meta)}: The base layer.
    \item \textbf{meta-poky}: Poky distribution policy.
    \item \textbf{meta-<bsp>}: Hardware support (e.g., meta-raspberrypi).
    \item \textbf{meta-<yourcompany>}: Your proprietary application and logic.
\end{itemize}

\section{Modifying Recipes with .bbappend}
\textbf{Never edit a recipe in an upstream layer directly.} Instead, use a \texttt{.bbappend} file in your own layer to override or extend it.

\subsection{Rules for bbappend}
\begin{enumerate}
    \item The filename must match the original recipe name exactly (using \texttt{\%} as a wildcard for version).
    \item Your layer must have a higher (or equal) priority.
    \item The original recipe must be in a layer present in \texttt{bblayers.conf}.
\end{enumerate}

\subsection{Example: Patching the Kernel}
To apply a patch to the Linux kernel recipe in \texttt{meta-bsp}:
Create \texttt{meta-custom/recipes-kernel/linux/linux-yocto\_\%.bbappend}:
\begin{lstlisting}
FILESEXTRAPATHS:prepend := "${THISDIR}/files:"

SRC_URI += "file://0001-fix-my-driver.patch"
\end{lstlisting}

\section{Layer Priority}
Layers define a priority in \texttt{conf/layer.conf}.
\begin{lstlisting}
BBFILE_PRIORITY_meta-custom = "10"
\end{lstlisting}
If a recipe exists in multiple layers, the one from the highest priority layer is used. \texttt{.bbappend} files are applied regardless of priority, but priority matters for masking.

\section{Hands-On: Creating a Custom Image}
Instead of modifying \texttt{local.conf} to add packages, create a dedicated image recipe.

Create \texttt{meta-custom/recipes-core/images/my-product-image.bb}:
\begin{lstlisting}
require recipes-core/images/core-image-base.bb

SUMMARY = "My Product Image"

IMAGE_INSTALL += "hello"
IMAGE_INSTALL += "openssh"
IMAGE_INSTALL += "htop"

# Add 100MB of free space
IMAGE_ROOTFS_EXTRA_SPACE = "102400"
\end{lstlisting}

Build it:
\begin{lstlisting}[language=bash]
bitbake my-product-image
\end{lstlisting}
