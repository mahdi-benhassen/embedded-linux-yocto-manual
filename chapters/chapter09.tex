\chapter{Yocto Project Fundamentals}

\section{Introduction}
The Yocto Project is an open-source collaboration project that provides templates, tools, and methods to help you create custom Linux-based systems for embedded products regardless of the hardware architecture. It is not a Linux distribution itself; rather, it is a tool to \emph{create} a Linux distribution.

\section{Terminology De-mystified}
Newcomers often confuse the following terms. Understanding the distinction is crucial:

\begin{description}
    \item[The Yocto Project] The umbrella organization and working group under the Linux Foundation.
    \item[OpenEmbedded (OE)] The build framework and community that maintains the metadata (recipes). Yocto is based on OpenEmbedded.
    \item[BitBake] The build engine (make-like tool). It parses metadata, constructs a dependency graph, and executes tasks.
    \item[Poky] The reference distribution of the Yocto Project. It is a combined repository containing BitBake, OpenEmbedded-Core (OE-Core), and Yocto-specific configuration.
    \item[Metadata] The files that define \emph{what} to build and \emph{how}. These include recipes (\texttt{.bb}), configuration files (\texttt{.conf}), classes (\texttt{.bbclass}), and includes (\texttt{.inc}).
\end{description}

\section{The Build Workflow}
The Yocto build process transforms source code into a bootable image through a pipeline of steps:

\begin{enumerate}
    \item \textbf{Fetch}: BitBake downloads source code from upstream repositories (Git, tarballs, SVN) into a local download directory (\texttt{DL\_DIR}).
    \item \textbf{Extract \& Patch}: Sources are unpacked into a working directory (\texttt{WORKDIR}) and patches are applied.
    \item \textbf{Configure}: The build system configures the software (e.g., running \texttt{./configure} or \texttt{cmake}).
    \item \textbf{Compile}: The source is compiled using the cross-toolchain.
    \item \textbf{Install}: Binaries and files are installed into a temporary destination directory (\texttt{D}).
    \item \textbf{Package}: Files in \texttt{D} are split into packages (RPM, DEB, or IPK).
    \item \textbf{Image Generation}: BitBake selects the required packages and installs them into a root filesystem image.
    \item \textbf{SDK Generation}: Optionally, a Software Development Kit (SDK) is generated for application developers.
\end{enumerate}

\section{The Build Directory Layout}
When you initialize a build environment (using \texttt{oe-init-build-env}), a build directory is created. Its structure is standardized:

\begin{itemize}
    \item \texttt{conf/}: Configuration files.
    \begin{itemize}
        \item \texttt{local.conf}: User-specific settings (machine, parallelism, disk space).
        \item \texttt{bblayers.conf}: Defines which layers are active in the build.
    \end{itemize}
    \item \texttt{tmp/}: The temporary work area. \textbf{This can be massive (50GB+)}.
    \begin{itemize}
        \item \texttt{tmp/work/}: Where compilation happens. Organized by architecture and recipe.
        \item \texttt{tmp/deploy/}: Final output (images, packages, SDKs).
        \item \texttt{tmp/sysroots/}: Shared libraries and headers for cross-compilation (deprecated in modern Yocto in favor of recipe-specific sysroots).
    \end{itemize}
    \item \texttt{downloads/} (\texttt{DL\_DIR}): The cache of downloaded source tarballs. Shared across builds.
    \item \texttt{sstate-cache/} (\texttt{SSTATE\_DIR}): The Shared State cache. Accelerates builds by reusing pre-built artifacts. \textbf{Critical for performance}.
\end{itemize}

\section{Shared State Cache (sstate)}
Yocto's "secret weapon" is the sstate cache. Before executing a task (like compilation), BitBake checks if a valid artifact exists in the sstate-cache. If the inputs (compiler version, source code, configuration) haven't changed, it simply extracts the result instead of rebuilding.

\begin{tip}
    Always configure a persistent \texttt{DL\_DIR} and \texttt{SSTATE\_DIR} outside your build directory to speed up future builds and save disk space.
\end{tip}
