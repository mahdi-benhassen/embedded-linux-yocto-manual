 \chapter{Linux Kernel Basics}
 \section{Configuration}
 Configure the kernel to match the target hardware using defconfig and menuconfig. Important areas include processor type, buses, filesystems, networking, and device drivers.
 \begin{lstlisting}[language=bash]
 git clone https://git.kernel.org/pub/scm/linux/kernel/git/stable/linux.git
 cd linux
 make ARCH=arm CROSS_COMPILE=arm-linux-gnueabihf- defconfig
 make ARCH=arm CROSS_COMPILE=arm-linux-gnueabihf- menuconfig
 \end{lstlisting}
 \section{Build and Install}
 Build the kernel, device tree blobs, and modules using cross-compilation.
 \begin{lstlisting}[language=bash]
 make ARCH=arm CROSS_COMPILE=arm-linux-gnueabihf- -j$(nproc) zImage dtbs
 make ARCH=arm CROSS_COMPILE=arm-linux-gnueabihf- modules
 \end{lstlisting}
 \section{Device Tree}
 The device tree describes non-enumerable hardware such as memory, buses, and peripherals. Adjust nodes and properties to match board schematics and verify driver bindings.
 \section{Debugging}
 Use earlyprintk, dynamic debug, and printk levels to inspect boot failures. Enable symbols and compile with debug options to improve observability when using gdb and perf.
