 \chapter{Development Environment Setup}
 \section{Host Requirements}
 A reliable host environment is required for cross-compilation and image creation. Use a recent Linux distribution with git, gcc, g++, make, python3, rsync, tar, gzip, bzip2, cpio, and qemu. Windows hosts can use WSL2 with an Ubuntu distribution.
 \section{Core Tools}
 Install git, compilers, and QEMU. The following commands target Debian-based systems.
 \begin{lstlisting}[language=bash]
 sudo apt update
 sudo apt install -y build-essential git python3 python3-pip \
   qemu-system-arm qemu-system-aarch64 qemu-system-x86 \
   cpio rsync bzip2 gzip unzip tar
 \end{lstlisting}
 \section{Version Control}
 Configure your identity and a clean workflow for embedded projects.
 \begin{lstlisting}[language=bash]
 git config --global user.name "Engineer"
 git config --global user.email "engineer@example.com"
 git config --global pull.rebase true
 git config --global init.defaultBranch main
 \end{lstlisting}
 \section{Serial and Networking}
 Interact with boards using serial consoles and SSH. On Linux, connect with tools like screen or picocom. Use DHCP or static IP for networking. Confirm connectivity with ping and ssh.
 \begin{lstlisting}[language=bash]
 sudo apt install -y screen picocom openssh-client
 screen /dev/ttyUSB0 115200
 ssh user@board-ip
 \end{lstlisting}
 \section{Editors and Build Integration}
 Use an extensible editor with support for cross-development. Pair an editor such as VS Code with remote SSH to edit files on target boards during debug. Integrate terminals for build, deployment, and flashing workflows.
 \section{Hands-On: QEMU Sanity Boot}
 Verify the environment by booting a prebuilt kernel and initramfs under QEMU.
 \begin{lstlisting}[language=bash]
 wget https://busybox.net/downloads/binaries/1.36.1-i686/busybox
 chmod +x busybox
 mkdir -p ~/devsetup/initramfs/bin
 cp busybox ~/devsetup/initramfs/bin
 cd ~/devsetup/initramfs
 ln -s bin/busybox sh
 echo '#!/bin/sh' > init
 echo 'mount -t proc none /proc' >> init
 echo 'mount -t sysfs none /sys' >> init
 echo 'echo "Environment OK"' >> init
 echo '/bin/sh' >> init
 chmod +x init
 find . | cpio -H newc -o | gzip > ../initramfs.cpio.gz
 \end{lstlisting}
 Launch QEMU with a valid kernel image and the generated initramfs.
 \begin{lstlisting}[language=bash]
 qemu-system-arm -M virt -cpu cortex-a9 -m 256M -nographic \
   -kernel path/to/zImage -initrd ~/devsetup/initramfs.cpio.gz \
   -append "console=ttyAMA0"
 \end{lstlisting}
